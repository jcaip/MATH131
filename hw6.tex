\documentclass[10pt,a4paper]{article}
\addtolength{\oddsidemargin}{-.875in}
\addtolength{\evensidemargin}{-.875in}
\addtolength{\textwidth}{1.75in}
\addtolength{\topmargin}{-.875in}
\addtolength{\textheight}{1.75in}

\usepackage{amsmath,amssymb}
\DeclareMathOperator*{\E}{\mathbb{E}}
\DeclareMathOperator*{\R}{\mathbb{R}}
\DeclareMathOperator*{\Q}{\mathbb{Q}}
\DeclareMathOperator*{\N}{\mathbb{N}}
\DeclareMathOperator*{\I}{\mathbb{I}}
\usepackage{mathtools}
\DeclarePairedDelimiter{\ceil}{\lceil}{\rceil}
\DeclarePairedDelimiter{\abs}{\lvert}{\rvert}
\begin{document}

MATH 131 Homework 6

Jesse Cai

304634445

\begin{enumerate}
    \item \textbf{Prove $\forall x \in \R: x \in A  = \left(\forall k \in \N: A_k := \{n \in \N  : \abs{a_n - x} < \frac{1}{k+1}\} \text{ is unbounded } \right)$}

        Fix $x \in A$. Then since $x$ is a subsequential limit, $\forall k \exists N \forall n : n > N \implies \abs{a_{n} - x} < \frac{1}{k+1}$. 


        Assume $A_k$ was bounded so there was a bound $M$.

        Then fix $k = M$ so that $\exists N \forall n : n > N \implies \abs{a_{n} - x} < \frac{1}{M+1}$. 

        But then we get $\forall n > N: \abs{a_{n} -x} > M+1 > M$, which is a contradiction, since we said the bound was $M$.

    \item \textbf{Prove $b, c \in A$.}

        We know from Theorem 11.7 in the book that There exists a monotonic subsequence whose limit is $\limsup s_n$ and $\liminf s_n$ respetively, so $b, c \in A$.

        \textbf{Assuming that $\lim (a_{n+1} - a_n) = 0$ prove that $A = [b,c]$.}

        From above, we know that $b,c$ are the min/max of the interval, and are included. 

        Take $x,y$ such that $b < x < y < c$. Then $A = \{n: a_n < x\}$ and $B = \{n: a_n > y\}$ are infinite. 

        Assume that $\{n : x<n<y\}$ was finite. Then we can find a sequence $n_k \in A, n_{k+1} \in B$, but this is a contradiction since time implies that $a_{n_{k+1}} - a_{n_k} > 0$, so this must be infinite.

        But since all three intervals are infinite $A$ is the whole closed interval.

    \item \textbf{Prove $( s_n )$ is Cauchy and hence convergent.}

        $ (s_n)$ is Cauchy if $\forall \epsilon > 0, \exists N, \forall m, n > N: \abs{s_m - s_n} < \epsilon$

        WLOG assume $m>n$.

        Then $\abs{s_m - s_n } = \abs{s_m - s_{m-1} + s_{m-1} - s_{m-2} + \cdots + s_{n+1} - s_n}$. So by triangle inequality 
        $\abs{s_m - s_n } \leq \abs{s_m - s_{m-1}} +\abs{s_{m-1} - s_{m-2}} + \cdots + \abs{s_{n+1} - s_n}$

        But we know $\forall n \in \N: \abs{s_{n+1} - s_n} < 2^{-n}$ so 

        $\abs{s_m - s_n } \leq 2^{-m+1} + 2^{-m+2} + \cdots + 2^{-n} = 2^{-m+!}$

        So for any $\epsilon$ choose $N = $ so $(s_n)$ is Cauchy and hence convergent.

        \textbf{Is this true if $\abs{s_{n+1} - s_n} < \frac{1}{n}$ }

        No, take $(s_n) = \sum_{i=1}{n} \frac{1}{i}$. Then $\abs{s_{n+1} - s_n} = \frac{1}{n+1} < \frac{1}{n}$ but $(s_n)$ is not Cauchy.

    \item \textbf{Consider the following sequences $$a_n = (-1)^n, b_n = \frac{1}{n}, c_n = n^2, d_n = \frac{6n+4}{7n-3}$$}

        Monotone subsequences: $(a_{2n}), (b_n), (c_n ), (d_n)$.

        Set of subsequential limits: $\{-1, 1\}, \{0\}, \{\infty \}, \{\frac{6}{7}\}$

        $\limsup$ and $\liminf$: $1,-1; 0, 0; \infty, \infty; \frac{6}{7}, \frac{6}{7}$

        $a_n$ does not converge, while $c_n$ diverges to $\infty$. $b_n, d_n$ both converge.

        $a_n, b_n, d_n$ are all bounded, while $c_n$ is unbounded.

    \item \textbf{Consider the following sequences $$w_n = (-2)^n, x_n = 5^{(-1)^n}, y_n = 1 + (-1)^n, z_n = n \cos \left(\frac{n\pi}{4} \right) $$}

        Monotone subsequences: $(w_{2n}), (x_{2n}), (y_{2n} ), (z_{4n-2})$.

        Set of subsequential limits: $\{-\infty, \infty\}, \{\frac{1}{5}, 5\}, \{0, 2\}, \{ -\infty, 0, \infty\}$

        $\limsup$ and $\liminf$: $-\infty, \infty; \frac{1}{5}, 5; 0,2 ; -\infty, \infty$

        None of these 4 sequences converge

        $w_n, z_n$ are bounded, while $x_n, y_n$ are unbounded.

    \item \textbf{Prove $\liminf s_n = -\limsup (-s_n)$}

        We know from Theorem 10.6 $\liminf s_n = \lim_{N \to \infty}  \inf \{ s_n: n > N\}$

        But from Exercise 5.6 we know that $\inf \{ s_n: n > N\} = - \sup \{-s_n : n > N\}$ so long as $S$ is a nonempty subseto of $\R$, but this is exactly what $\{ s_n: n > N\}$ is.

        So $\liminf s_n = \lim_{N \to \infty} - \sup \{-s_n : n > N\} = - \limsup (-s_n)$.
        
    \item \textbf{Determine which of the following series converge.}

        $\sum \frac{n-1}{n^2}$ Note that $n> 10 \implies frac{n-1}{n^2} > \frac{1}{2n}$, a divergent harmonic series, so this sum diverges by the comparison test.

        $\sum (-1)^n$ Note that $a_n = (-1)^n$ so $\lim a_n \neq 0 \implies$ divergence.

        $\sum \frac{3n}{n^3}  = 3 \sum \frac{1}{n^2}$ but this is a convergent harmonic series, so this sequence also converges. 

        $\sum {\frac{n^3}{3^n}}$. By the Ratio test, $\lim \abs{\frac{(n+1)^3}{3n^3}} = \frac{1}{3} < 1$. so this series converges.

        $\sum \frac{n^2}{n!}$. By the Ratio test, $\lim \abs{\frac{1}{n} + \frac{1}{n^2}} = 0 < 1$ so this series converges.
    
        $\sum \frac{1}{n^n}$. By the Root test, $\lim \abs{a_n}^{\frac{1}{n}} = \lim \abs{\frac{1}{n^n}}^{\frac{1}{n}} = \lim \frac{1}{n} = 0 < 1$ so this series converges.

        $\sum \frac{n}{2^n}$. By the Ratio test, $\lim \abs{\frac{n+1}{2n}} = \frac{1}{2} < 1$ so this series converges.

    \item \textbf{Prove that if $\sum \abs{a_n}$ converges and $(b_n)$ is bounded then $\sum a_n b_n$ converges.}

        We just need to show that $\sum a_n b_n$ is Cauchy to show that it converges. 
        
        Fix $\epsilon$. Then for $\frac{\epsilon}{M} \exists N ,  \forall m,n : m \geq n > N \implies  \abs{\sum_{k=n}^m a_k} < \frac{\epsilon}M$ since $\sum \abs{a_n}$ is Cauchy.
        
        Since $(b_n)$ is bounded, $\exists M > 0: \forall n \abs{b_n} \leq M$. 

        So $\abs{\sum_{k=n}^m a_k b_k } \leq M \sum_{k=n}^m \abs{a_k}$. But then multiplying by $M$ on both sides of the equality yields

        $\abs{\sum_{k=n}^m a_k b_k } \leq M \sum_{k=n}^m \abs{a_k} \leq M \frac{\epsilon}{M} = \epsilon$

        So $\sum a_n b_n$ is Cauchy and therefore converges.

    \item \textbf{Show if $\sum a_n$ and $\sum b_n$ are convergent series then $\sum \sqrt{a_n b_n}$ converges.}

        Since $a_n, b_n > 0 \implies a_n b_n \leq a_nb_n + (a_nb_n + a_n^2 + b_n^2) = (a_n + b_n)^2$

        This means that $\sqrt{a_n b_n} \leq a_n +b_n \implies \sum \sqrt{a_n + b_n} \leq \sum a_n + b_n$.

        But $\sum a_n +b_n$ is convergent, as it is the sum of two convergent series, so by the comparison test  $\sum \sqrt{a_n b_n}$ converges as well.

    \item \textbf{Prove there is a subsequence such that $\sum (a_{n_k})$ converges.}

        By Theorem 11.7 we know that there exists a monotonic subsequence $(a_{n_k})$ whose limit is $\liminf a_n = 0$.

        $\lim \abs{a_{n_k}} = 0 \implies \exists L_k \in \N, \forall s > L_k : \abs{a_{n_k} - 0} < \frac{1}{L_k^2}$

        If we take $k_l > \max(k_{l-1}+1, L_k) \implies \abs{a_{n_{k_l}}} < \frac{1}{l^2} \land k_l > k_{l-1}$. 
        
        Then we can use the comparison test to show that $\sum \abs{a_{n_{k_l}}}$ converges, as it is always less than a convergenet series, the second-order harmonic sum.


    \item \textbf{Predicate Calculus}

        Let $m\vert n = \exists k \in \N: (k)(n) = m$

        \begin{enumerate}
            \item $\forall n \in \N  : [(n \vert 3 ) \land (n \vert 2)] \implies (n \vert 7)$
            \item $\exists n \in \N  : (n \vert 6 ) \land \lnot (n \vert 4) $
            \item $\forall n \in \N  : [(n \vert 6 ) \land (n \vert 5)] \implies (n \vert 20)$
            \item $\exists n \in \N  : [(n \vert 3 ) \land (n \vert 2)] \land \lnot (n \vert 7)$
            \item $\forall n \in \N  : \not (n \vert 6 ) \lor (n \vert 4) $
            \item $\exists n \in \N  : [(n \vert 6 ) \land (n \vert 5)] \land \lnot (n \vert 20)$
        \end{enumerate}

\end{enumerate}
\end{document}
