\documentclass[10pt,a4paper]{article}
\addtolength{\oddsidemargin}{-.875in}
\addtolength{\evensidemargin}{-.875in}
\addtolength{\textwidth}{1.75in}

\addtolength{\topmargin}{-.875in}
\addtolength{\textheight}{1.75in}

\usepackage{amsmath,amssymb}
\DeclareMathOperator*{\E}{\mathbb{E}}
\DeclareMathOperator*{\R}{\mathbb{R}}
\DeclareMathOperator*{\Q}{\mathbb{Q}}
\DeclareMathOperator*{\N}{\mathbb{N}}
\DeclareMathOperator*{\I}{\mathbb{I}}
\begin{document}

MATH 131 Homework 2

Jesse Cai

304634445

\begin{enumerate}
    \item \textbf{Prove that addition is associative.}

        We need to show $\forall a,b,c \in \N:  (a+b)+c = a+(b+c)$. 

        We will use the definition of addition: $\forall n, m \in \N: n+S(m) = S(n+m)$, as well as P5 of the Peanno axioms.
        Let $a,b$ be fixed and $P(n) = (a +b)+n = a+ (b+n)$.

        Base case: $P(0)$ is TRUE by inspection.
        $$(a+b)+0 = a+b = a+(b) = a+(b+0)$$

        Inductive step: Assume $P(n)$ is true, then consider $P(S(n))$
        $$ (a+b) + S(n) = S((a+b)+n) =_{by P(n)} S(a+(b+n))  = a + S(b+c) = a  + (b + S(n))$$
        so $P(n) \implies P(S(n))$ and by P5 $\forall n \in \N P(n)$ holds.


    \item \textbf{Prove $\forall m, n, r\in \N : m \leq n \implies r \cdot m \leq r \cdot n$.}

        Let $n, m$ be fixed and $P(r)  = m \leq n \implies r \cdot m \leq r \cdot $
        Base case (r = 0):
            If $n > m$ the implication holds be default otherwise $0 \cdot m = 0 = 0 \cdot n$ so this is TRUE.

        Inductive step: Assume $P(r)$ is true, then consider $P(S(r))$
        $$m \leq n \implies S(r) \cdot m \leq S(r) \cdot n$$
        $$m \leq n \implies r \cdot m + m \leq r\cdot n + n$$

        Either $m \leq n$ in which case $r \cdot m \leq r\cdot n \implies r \cdot m + m \leq r\cdot n + n$ or $m > n$, in which case the $P(S(r))$ is true by default. 

        so $P(r) \implies P(S(r))$ and by P5 $\forall r \in \N P(r)$ holds.


    \item \textbf{Show $\sqrt[3]{5 - \sqrt3}$ is not rational.}

        Assume that $\sqrt[3]{5 - \sqrt3} \in \Q $.  Then, $\left(\sqrt[3]{5 - \sqrt3}\right)^3 \in \Q$, since the rationals are closed under multiplication.
        This means that $5 - \sqrt3 \in \Q$. But since $\Q$ is closed under subtraction $5 - (5 - \sqrt3) \in \Q \implies \sqrt3 \in \Q$ which is a contradiction of the RRT. Therefore $\sqrt[3]{5 - \sqrt3} \not \in \Q$.
    \item \textbf{Prove (v) $0<1$ and (vii) $\forall a,b, c \in \R 0 < a < b \implies 0 < b^{-1} < a^{-1}$}
    
        To prove $0<1$ we will use (iv). Let $a = 1$, then by (iv) $0 < 1^2 \implies 0 < 1$

        To prove (vii), notice by (vi) we get $ 0 < b^{-1} $ and $0 < a^{-1}$, so we just need to show that $b^{-1} < a^{-1}$. 

        By (iii) we get $ 0 < a^{-1} b^{-1}$ and we can use (i) to get $ -a^{-1}b^{-1} < -0$
        We can apply (ii) with $ c = -a^{-1}b^{-1}$ to get $bc \leq ac $ so 
        $ - a^{-1} \leq -b{-1}$. But then we can apply (i) again to get $b^{-1} \leq a^{-1}$ 

    \item \textbf{Prove $\forall a, b, c \in \R \lvert a+b+c\rvert \leq \lvert a \rvert+\lvert b \rvert+\lvert c \rvert $}

        Because addition is commutative $a+b+c = (a+b)+c \implies \lvert a+b+c\rvert = \lvert (a+b)+c\rvert$
        By triangle inequality we get $\lvert(a+b)+c\rvert \leq \lvert a+b\rvert + \lvert c \rvert$ so 
        $$
            \lvert a+b+c\rvert \leq \lvert a+b\rvert + \lvert c \rvert
        $$
        Again by triangle inequality,  $\lvert a+b\rvert \leq \lvert a\rvert + \lvert b \rvert$ so 
        $$
            \lvert a+b+c\rvert \leq \lvert a+b\rvert + \lvert c \rvert \leq \lvert a\rvert + \lvert b\rvert + \lvert c\rvert
        $$

        \textbf{Prove $\lvert a_1 + a_2 + \ldots + a_n \rvert \leq \lvert a_1 \rvert +\lvert a_2 \rvert +\ldots + \lvert a_n \rvert $}

        Let $P(n) = \lvert a_1 + a_2 + \ldots + a_n \rvert \leq \lvert a_1 \rvert +\lvert a_2 \rvert +\ldots + \lvert a_n \rvert$

        Base case: n=0 is trivial $\lvert emtpy \rvert$ = nothign 

        Inductive step: Assume $P(n)$ is TRUE, then consider $P(n+1)$

        $$
            \lvert a_1 + a_2 + \ldots +a_n + a_{n+1} \rvert  = 
            \lvert (a_1 + a_2 + \ldots +a_n) + a_{n+1} \rvert \leq
            \lvert a_1+ a_2+\ldots +a_n \rvert + \lvert a_{n+1}\rvert = 
            \lvert a_1 \rvert +\lvert a_2 \rvert +\ldots + \lvert a_n \rvert + \lvert a_{n+1}\rvert
        $$

        So $P(n) \implies P(n+1)$ and $\forall n \in \N P(n)$ holds.
 


    \item  \textbf{Prove $ \inf S \leq \sup S $}

        By definition $\forall x \in S. \sup S > x > \inf S$ so $\inf S \leq \sup S$.

        \textbf{What can you say about $S$ if $\inf S = \sup S$.}
        
        $S$ is just one element.

    \item \textbf{Give an example of $S$ and $T$ where $S \cap T \neq \emptyset$.} 
        
        Let $S = \{ 1 \} $ and $T = \{1, 2\}$ then $S \cap T = \{ 1 \} \neq \emptyset$

        \textbf{Give an example of $S$ and $T$ where $S \cap T = \emptyset$ and $\sup S = \inf T$.} 

        Let $S = \{ r \in \Q : r^2 < 7 \} $ and $T = \{ r \in \Q : r^2 > 7 \}$ then $S \cap T = \emptyset$ and $\sup S = \inf T = \sqrt7$.

    \item \textbf{Prove $a<b \implies \exists x \in I: a<x<b$}

        First we show $\{ r + \sqrt2 : r \in \Q\} \subset I$. By RRT we know that $\sqrt2 \not\in \Q$.

        Let $x \in \{ r + \sqrt2 : r \in \Q\}$. Suppose that $x = (r+i) \in \Q$ , then $(r+i) - r = i \in \Q$, since $\Q$ is closed under subtraction, but this is a contradiction of RRT, so therefore $x \not\in \Q \implies x \in I$.

        Now for any $a < b$ we can find a $r \in \Q : a - \sqrt2 < r < b - \sqrt2$, by the denseness of $\Q$ (Theorem 4.7). 
        But then if we take $r + \sqrt2$ we get $a < r+\sqrt2 < b$ and we showed $r+\sqrt2 \in I$, so there will always exists an irrational number $a < i < b$.

    \item \textbf{Prove $\sup(A+B) = \sup A + \sup B $.}

        To show $\sup(A+B) = \sup A + \sup B $, we will first show $\sup A + \sup B \geq\sup(A+B)$.

        Let $x \in A+B : x =a+b  \leq \sup A  + \sup B$ since $\forall a \in A. a \leq \sup A$ and likewise $\forall b \in B. b \leq \sup B$.  So $\sup (A+B) \leq \sup A +  \sup B$.

        Consider the set $A+\sup B = \{\forall a \in A: a+\sup B\}$. Clearly $A + \sup B \subset A+B $ so taking the suprema yields $\sup(A+\sup B)\leq \sup(A+B)$. 

        But $\sup(A + \sup B) = \sup A + \sup B \leq \sup(A+B)$. So since 
        $$\sup A + \sup B \leq \sup(A+B) \land \sup A + \sup B \geq \sup(A+B) \implies \sup A + \sup B = \sup(A+B)$$.

        \textbf{Prove $\inf(A+B) = \inf A + \inf B $.}

        To show $\inf(A+B) = \inf A + \inf B $, we will first show $\inf A + \inf B \leq \inf(A+B)$.

        Let $x \in A+B : x =a+b  \geq \inf A  + \inf B$ since $\forall a \in A. a \geq \inf A$ and likewise $\forall b \in B. b \geq \inf B$.  So $\inf (A+B) \geq \inf A +  \inf B$.

        Consider the set $A+\inf B = \{\forall a \in A: a+\inf B\}$. Clearly $A + \inf B \subset A+B $ so taking the suprema yields $\inf(A+\inf B)\geq \inf(A+B)$. 

        But $\inf(A + \inf B) = \inf A + \inf B \geq \inf(A+B)$. So since 
        $$\inf A + \inf B \leq \inf(A+B) \land \inf A + \inf B \geq \inf(A+B) \implies \inf A + \inf B = \inf(A+B)$$.





\end{enumerate}
\end{document}
