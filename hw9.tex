\documentclass[10pt,a4paper]{article}
\addtolength{\oddsidemargin}{-.875in}
\addtolength{\evensidemargin}{-.875in}
\addtolength{\textwidth}{1.75in}
\addtolength{\topmargin}{-.875in}
\addtolength{\textheight}{1.75in}

\usepackage{amsmath,amssymb}
\DeclareMathOperator*{\E}{\mathbb{E}}
\DeclareMathOperator*{\R}{\mathbb{R}}
\DeclareMathOperator*{\Q}{\mathbb{Q}}
\DeclareMathOperator*{\N}{\mathbb{N}}
\DeclareMathOperator*{\I}{\mathbb{I}}
\usepackage{mathtools}
\DeclarePairedDelimiter{\ceil}{\lceil}{\rceil}
\DeclarePairedDelimiter{\abs}{\lvert}{\rvert}
\begin{document} 
MATH 131 Homework 9

Jesse Cai

304634445

\begin{enumerate}
    \item \textbf{Let $f(x) = x^{\frac{1}{3}}$ show $f'(x) = \frac{1}{3}x^\frac{-2}{3}$}

        For $a \neq 0$:

        $$f'(x) = \lim_{x \to a} \frac{f(x) - f(a)}{x - a } =\lim_{x \to a} \frac{x^{\frac{1}{3}} - a^{\frac{1}{3}}}{x - a} = \lim_{x \to a} \frac{x^{\frac{1}{3}} - a^{\frac{1}{3}}}{(x^{\frac{1}{3}} - a^{\frac{1}{3}})(x^{\frac{2}{3}} + x^{\frac{1}{3}}a^{\frac{1}{3}} + a^{\frac{2}{3}})} = \lim_{x \to a} \frac{1}{x^{\frac{2}{3}} + x^{\frac{1}{3}}a^{\frac{1}{3}} + a^{\frac{2}{3}}} =  \frac{1}{3}a^{-\frac{2}{3}}$$

    \item \textbf{Let $f(x)=x^2$ rational and $f(x)=0$ irrational.}
        
    \textbf{ Prove $f$ is continuous at $x = 0$}

        $f$ is continuous at $0$ if $\forall \delta 0 \forall x \in \R \exists \epsilon> 0: \abs{x - 0 }< \delta \implies \abs{f(x) -f(0)} <\epsilon$

        When $x \in \Q: \abs{f(x) -f(0)} = \abs{x^2 -0} < \abs{x^2}$, and $x \not \in \Q: \abs{f(x) -f(0)} = \abs{0 - 0} = 0 $ so $\abs{f(x)-f(0)} < \abs{x}^2$.

        Fix $\delta$. Take $\epsilon = \sqrt\delta$. 

        Then $\abs{x-0} < \delta \implies \abs{x} < \delta \implies \abs{x}^2 < \delta^2$.

        But then $\abs{f(x)-f(0)} < \abs{x}^2 = \delta^2 =  \epsilon$, so $f$ is continuous at 0.
    
    \textbf{ Prove $f$ is not continuous $\forall x \neq 0$}

        Pick $a \neq 0$. 

        WLOG If $a$ is irrational then there exists a sequence of rational numbers that converges to $a$. but then $f(a_n) = a^2 \implies \lim f(a_n) = a^2 \neq 0$. But $f(x) = 0 $ by definition so $f$ is discontinuous at $a$.

    \textbf{ Prove $f$ is differentiable at $x = 0$.}

        $$f'(0)= \lim_{x \to 0} \frac{f(x) - f(0)}{x - 0 } = \lim_{x \to 0} \frac{f(x) - 0}{x - 0 } = \lim_{x \to 0} \frac{f(x)}{x}$$

        When $x \in \Q: \lim_{x\to 0} \frac{f(x)}{x} = \lim_{x \to 0} x = 0$.

        Simlarly when $x \not \in \Q: \lim_{x\to 0} \frac{0}{x} = 0$.

        So $\lim_{x \to 0} \frac{f(x)}{x} = 0 = f'(0)$.
    
    \item \textbf{Prove if $f$ and $g$ are differentiable and $f(0)=g(0)$ and $\forall x : f'(x) \leq g'(x)$ then $\forall x \geq 0 : f(x) < g(x)$.}

        Consider $h(x) = f(x) -g(x)$. Then since $f'(x) \leq g'(x) \implies  h'(x) = f'(x) - g'(x) \leq 0$, and $h(x)$ is differentiable on $\R$. 

        Fix $x$. By Mean Value Theorem $\exists y \in (0, x): h'(x) = \frac{h(x) - h(0)}{x-0}$. 

        But then $ h(x)- h(0) < h'(x) < 0 \implies h(x) < h(0)$.

        But then $f(x) - g(x)  < f(0) - g(0) \implies f(x) \leq g(x)$.

    \item \textbf{Show $\forall x \in (0, \frac{\pi}{2}) x < \tan x $}
        Let $f(x) = x - \tan x$. Then $f'(x) = 1 - \sec^2 x = 1 - (1 + \tan^2 x) = \tan^2 x$, which is $>0$ for all $x \in (0, \frac{\pi}{2})$. So $f$ is strictly increasing and note $f(0) = 0 - \tan 0 = 0$ so $\forall x \in (0, \frac{\pi}{2} : f(x) > 0  \implies x < \tan x$

    \textbf{Show that $\frac{x}{\sin x}$ is strictly increasing. }

        Let $f(x) = \frac{x}{\sin x}$, then $f'(x) = \frac{\sin x - x \cos x }{\sin^2 x}$. 

        But from above we know $x < \tan x \implies \sin x > x \cos x$. So $\forall x \in (0, \frac{\pi}{2}: f'(x) > 0 $, which means $f(x)$ is strictly increasing.

    \textbf{Show that $\forall x \in [0, \frac{\pi}{2}]: x \leq \frac{\pi}{2} \sin x$}

        Note that at the endpoints $x = \frac{\pi}{2} \sin x$ and $\frac{x}{\sin x}$ is stricty increasing as shown above.,

        $$\forall x, y: 0 < x < y< \frac{\pi}{2} \implies \frac{x}{\sin x} < \frac{y}{\sin y} < \lim_{y \to \frac{\pi}{2}} \frac{y}{\sin y} = \frac{\pi}{2}$$

    \item \textbf{Let $f$ be differentiable on $\R$ with $a = \sup \{ \abs{f'(x)} : x \in \R \} < 1$.Show $(s_n)$ converges}
        
        Note $f$ is continuous and differentiable on $\R$, so we can use the MVT on the interval $(s_{n-1}, s_n)$ to know that $\exists y \in (s_{n-1}, s_n) : f'(y) = \frac{f(s_n) - f(s_{n-1})}{s_n - s_{n-1}}$.

        But then using the fact that $a < 1 $ we get 

        $$\abs{\frac{f(s_n) - f(s_{n-1})}{s_n - s_{n-1}}} \leq  \implies \abs{f(s_n) - f(s_{n-1})} \leq a \abs{s_n - s_{n-1}} \implies \abs{s_{n+1} -s_n} \leq a \abs{s_n - s_{n-1}}$$

        But we can apply this recursive defintion to get $\abs{s_{n+1} -s_n} \leq a^n \abs{s(1)-s(0)}$.

        $(s_n)$ converges if $\forall \epsilon > 0 \exists N \forall m > n > N: \abs{s_m - s_n} < \epsilon$

        Fix $s(0) \in \R, \epsilon>0$. Then we pick.
    \textbf{ Prove $f(x)= x$ has a fixed point.}

        From above, we know $(s_n)$ converges, so let $s = \lim s_n$. Since $(s_n)$ converges, so does $(f(s_n))$, also to $s$, as it is the same sequences, just delayed.

        Then because $f$ is differentiable and continuous on $\R$ $\lim s_n = s \implies \lim f(s_n)= f(s) \implies f(s) = s$ and $f$ has a fixed point.

    \item \textbf{Find $\lim_{x \to 0} \frac{x^3}{\sin x - x}$}

        Since $\lim x^3 = \lim g(x) = 0$ we can apply LHopital's rule.

        $\lim_{x \to 0} \frac{x^3}{\sin x - x} = \lim_{x \to 0} \frac{3x^2}{\cos x - 1} = \frac{0}{0}$

        Note again $\lim x^2 = \lim \cos x -1 = 0$ so we can apply LHoptial's rule twice again.

        $$\lim_{x \to 0} \frac{x^3}{\sin x - x} = \lim_{x \to 0} \frac{3x^2}{\cos x - 1} = \lim_{x \to 0} \frac{6x}{-\sin x} = \lim_{x \to 0} \frac{6}{-\cos x} = -6$$

    \textbf{Find $\lim_{x \to 0} \frac{\tan x - x}{x^3}$}

        Again we recursively apply LHopitals rule

        $$\lim_{x \to 0} \frac{\tan x -x}{ x^3} = \lim_{x \to 0}\frac{\sec^2 -1}{3x^2} = \lim_{x \to 0} \frac{2 \tan x \sec^2 x }{6x} = \lim_{x \to 0} \frac{2 \sec^2 x + 6 \tan^2 x \sec^2 x}{6} = \frac{1}{3}$$

    \textbf{Find $\lim_{x \to 0} \frac{1}{\sin x } - \frac{1}{x}$}

        We can rewrite this as $\lim_{x \to 0} \frac{x - \sin x}{ x \sin x }$, then apply LHopital's rule.
        $$\lim_{x \to 0} \frac{x - \sin x}{ x \sin x } = \lim_{x \to 0} \frac{1 - \cos x}{\sin x +  x \cos x } = \lim_{x \to 0} \frac{\sin x}{2 \cos x - x \sin x } = 0$$ 

    \item \textbf{Find $\lim_{x \to 0} (1 + 2x)^{\frac{1}{x}}$.}
        
        We can rewrite $(1 + 2x)^{\frac{1}{x}} =  e^{\frac{\log 1 + 2x}{x}}$ so $\lim_{x \to 0} (1 + 2x)^{\frac{1}{x}} = \lim_{x \to 0} e ^{\frac{\log 1 + 2x}{x}} = e^{\lim_{x \to 0}{\frac{\log 1 + 2x}{x}}} $.

        Then by LHopital's rule:

        $$\lim_{x \to 0}{\frac{\log 1 + 2x}{x}} =  \lim_{x \to 0}{\frac{2}{1+2x}} = 2 \implies \lim_{x \to 0} (1 + 2x)^{\frac{1}{x}} = e^2 $$

    \textbf{ Find $\lim_{y \to \infty} (1 + 2/y)^{y}$ }

        We can rewrite $(1 + 2/y)^{y}$ as $e^\frac{\log(1+2/y)}{y^(-1)}$

        Then by LHopital's rule: $\lim_{y \to \infty}\frac{\log 1 + \frac{2}{y}}{\frac{1}{y}} = \frac{2}{1 + \frac{2}{y}} = 2$

    \item \textbf{Let $f$ be differentiable on some interval $(c, \infty)$ and suppose $\lim_{x \to \infty} [f(x) + f'(x)]$. Prove $\lim_{x \to \infty} f(x) = L$ and $\lim_{x \to \infty} f'(x) = 0$.}

        Note that $f(x) = \frac{f(x) e^x}{e^x}$. Then by LHopital's rule

        $$\lim_{x \to \infty} f(x) = \lim_{x \to \infty} \frac{f(x) e^x}{e^x} = \lim_{x \to \infty} \frac{f(x) e^x + f'(x)e^x}{e^x} = \lim_{x \to \infty} \frac{[f(x) + f'(x)]e^x}{e^x} = L$$

        But then if $\lim_{x \to \infty} f(x) = L$ by the sum rule of limits $\lim_{x \to \infty} f(x) + f'(x) = \lim f(x) + \lim f'(x)$ so $\lim f'(x) = 0$.


\end{enumerate}
\end{document}
