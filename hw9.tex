\documentclass[10pt,a4paper]{article}
\addtolength{\oddsidemargin}{-.875in}
\addtolength{\evensidemargin}{-.875in}
\addtolength{\textwidth}{1.75in}
\addtolength{\topmargin}{-.875in}
\addtolength{\textheight}{1.75in}

\usepackage{amsmath,amssymb}
\DeclareMathOperator*{\E}{\mathbb{E}}
\DeclareMathOperator*{\R}{\mathbb{R}}
\DeclareMathOperator*{\Q}{\mathbb{Q}}
\DeclareMathOperator*{\N}{\mathbb{N}}
\DeclareMathOperator*{\I}{\mathbb{I}}
\usepackage{mathtools}
\DeclarePairedDelimiter{\ceil}{\lceil}{\rceil}
\DeclarePairedDelimiter{\abs}{\lvert}{\rvert}
\begin{document} 
MATH 131 Homework 9

Jesse Cai

304634445

\begin{enumerate}
    \item \textbf{Let $f(x) = x^{\frac{1}{3}}$ show $f'(x) = \frac{1}{3}x^\frac{-2}{3}$}

        For $a \neq 0$:

        $$f'(x) = \lim_{x \to a} \frac{f(x) - f(a)}{x - a } =\lim_{x \to a} \frac{x^{\frac{1}{3}} - a^{\frac{1}{3}}}{x - a} = \lim_{x \to a} \frac{x^{\frac{1}{3}} - a^{\frac{1}{3}}}{(x^{\frac{1}{3}} - a^{\frac{1}{3}})(x^{\frac{2}{3}} + x^{\frac{1}{3}}a^{\frac{1}{3}} + a^{\frac{2}{3}})} = \lim_{x \to a} \frac{1}{x^{\frac{2}{3}} + x^{\frac{1}{3}}a^{\frac{1}{3}} + a^{\frac{2}{3}}} =  \frac{1}{3}a^{-\frac{2}{3}}$$

    \item \textbf{Let $f(x)=x^2$ rational and $f(x)=0$ irrational.}
        
    \textbf{ Prove $f$ is continuous at $x = 0$}

        $f$ is continuous at $0$ if $\forall \delta 0 \forall x \in \R \exists \epsilon> 0: \abs{x - 0 }< \delta \implies \abs{f(x) -f(0)} <\epsilon$

        When $x \in \Q: \abs{f(x) -f(0)} = \abs{x^2 -0} < \abs{x^2}$, and $x \not \in \Q: \abs{f(x) -f(0)} = \abs{0 - 0} = 0 $ so $\abs{f(x)-f(0)} < \abs{x}^2$.

        Fix $\delta$. Take $\epsilon = \sqrt\delta$. 

        Then $\abs{x-0} < \delta \implies \abs{x} < \delta \implies \abs{x}^2 < \delta^2$.

        But then $\abs{f(x)-f(0)} < \abs{x}^2 = \delta^2 =  \epsilon$, so $f$ is continuous at 0.
    
    \textbf{ Prove $f$ is not continuous $\forall x \neq 0$}

        Pick $a = \sqrt 20$.

    \textbf{ Prove $f$ is differentiable at $x = 0$.}

        $$f'(0)= \lim_{x \to 0} \frac{f(x) - f(0)}{x - 0 } = \lim_{x \to 0} \frac{f(x) - 0}{x - 0 } = \lim_{x \to 0} \frac{f(x)}{x}$$

        When $x \in \Q: \lim_{x\to 0} \frac{f(x)}{x} = \lim_{x \to 0} x = 0$.

        Simlarly when $x \not \in \Q: \lim_{x\to 0} \frac{0}{x} = 0$.

        So $\lim_{x \to 0} \frac{f(x)}{x} = 0 = f'(0)$.
    
    \item \textbf{Prove if $f$ and $g$ are differentiable and $f(0)=g(0)$ and $\forall x : f'(x) \leq g'(x)$ then $\forall x \geq 0 : f(x) < g(x)$.}

        Consider $h(x) = f(x) -g(x)$. Then since $f'(x) \leq g'(x) \implies  h'(x) = f'(x) - g'(x) \leq 0$, and $h(x)$ is differentiable on $\R$. 

        Fix $x$. By Mean Value Theorem $\exists y \in (0, x): h'(x) = \frac{h(x) - h(0)}{x-0}$. 

        But then $ h(x)- h(0) < h'(x) < 0 \implies h(x) < h(0)$.

        But then $f(x) - g(x)  < f(0) - g(0) \implies f(x) \leq g(x)$.

    \item \textbf{Show $\forall x \in (0, \frac{\pi}{2}) x < \tan x $}
        Let $f(x) = x - \tan x$. Then $f'(x) = 1 - \sec^2 x = 1 - (1 + \tan^2 x) = \tan^2 x$, which is $>0$ for all $x \in (0, \frac{\pi}{2})$.

        Then by MVT $\foralll x \exists y \in (0, x): f'(x) = \frac{f(x) - f(0)}{x-0}$

    \item \textbf{Placeholder}
    \item \textbf{Find $\lim_{x \to \infty} \frac{x - \sin x}{x}$}
        $$\lim_{x \to \infty} \frac{x - \sin x}{x} = \lim _{x \to \infty} \frac{x}{x} - \frac{\sin x}{x} = \lim_{x \to \infty} 1 - \lim_{x \to \infty} \frac{\sin x}{x} = 1 - 0 = 1$$
    \item \textbf{Placeholder}
    \item \textbf{Placeholder}

\end{enumerate}
\end{document}
