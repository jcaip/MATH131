\documentclass[10pt,a4paper]{article}
\addtolength{\oddsidemargin}{-.875in}
\addtolength{\evensidemargin}{-.875in}
\addtolength{\textwidth}{1.75in}
\addtolength{\topmargin}{-.875in}
\addtolength{\textheight}{1.75in}

\usepackage{amsmath,amssymb}
\DeclareMathOperator*{\E}{\mathbb{E}}
\DeclareMathOperator*{\R}{\mathbb{R}}
\DeclareMathOperator*{\Q}{\mathbb{Q}}
\DeclareMathOperator*{\N}{\mathbb{N}}
\DeclareMathOperator*{\I}{\mathbb{I}}
\usepackage{mathtools}
\DeclarePairedDelimiter{\ceil}{\lceil}{\rceil}
\DeclarePairedDelimiter{\abs}{\lvert}{\rvert}
\begin{document} 
MATH 131 Homework 7

Jesse Cai

304634445

\begin{enumerate}
    \item \textbf{Find a series which diverges by the Root Test but for which the Ratio Test gives no information.}

        Consider the series $\sum a_n = \sum (2+ (-1)^n)^n$. Then the ratio test $\abs{\frac{a_{n+1}}{a_n}} = \abs{\frac{(2+ (-1)^{n+1})^{n+1}}{(2+ (-1)^{n})^{n}}}$

        If $n=2k$ is even then $\abs{\frac{a_{n+1}}{a_n}} = \abs{\frac{(2 + -1)^{2k+1}}{(3)^{2k}}} = \abs{\frac{1}{3^{2k}}}$ which converges to 0.

        If $n=2k+1$ is odd then $\abs{\frac{a_{n+1}}{a_n}} = \abs{\frac{(2 + 1)^{2k+2}}{(2-1)^{2k+1}}} = \abs{3^{2k+2}}$ which diverges to $\infty$.

        So the set of subsequential limits is $\{0, \infty \}$ and so $\liminf \abs{\frac{a_{n+1}}{a_n}} = 0 < 1 < \limsup \abs{\frac{a_{n+1}}{a_n}} = \infty$ so the Ratio test gives no information.

        The Root test $\abs{a_n}^{\frac{1}{n}} =  2 + (-1)^n$. So if $n$ is even this is 3 and if $n$ is odd this is 1.
        Thus $\limsup \abs{a_n}^{\frac{1}{n}} = 3 > 1$ so this series diverges.

    \item \textbf{ Prove that $\sum_1^\infty \frac{1}{n^2}$ converges by comparing it to $\sum_1^\infty \frac{1}{n(n+1)}$}

        Note that $a_n < b_n$ and $b_n = \frac{(n+1) - n }{n(n+1)} = \frac{1}{n} - \frac{1}{n+1}$ but then $\lim b_n = \lim (\frac{1}{n} - \frac{1}{n+1}) = \lim \frac{1}{n} - \lim \frac{1}{n+1} = 0$ so $\{ b_n \}$ is Cauchy and thus $\sum b_n$ converges.

        But then by the Comparison Test  $n(n+1) =  n^2 +n \geq n \implies \abs{a_n} < b_n \implies \sum a_n \leq \sum b_n$ so $\sum a_n$ converges as well.

            
    \item \textbf{Give an example of a divergent series $\sum a_n$ for which $\sum a_n^2$ converges.}

        Take the series $a_n = \frac{1}{n}$. This is the first harmonic, which diverges, but $\sum a_n^2 = \sum \frac{1}{n^2}$ is the second harmonic which converges.

        \textbf{Observe that if $\sum a_n$ is a convergent series of nonnegative terms then $\sum a_n^2$ converges.}

        We know that for $\sum a_n$ to be convergent, $\lim a_n = 0$ so for $\epsilon = 1 \exists N \forall n > N : \abs{a_n} < 1 \implies \forall n > N: \abs{a_n^2} < a_n$.  But then via the comparison test $\sum a_n^2$ converges as well.
            
        \textbf{Give an example of a convergent series $\sum a_n$ for which $\sum a_n^2$ diverges.}

        Consider the series $a_n = \frac{(-1)^n}{\sqrt n }$. This series converges due to the Alternating Series Theorem, as $\lim \frac{1}{\sqrt n } = 0$. 
        But $\sum a_n^2 = \sum \frac{1}{n}$ which is the first harmonic series and diverges.

    \item \textbf{Prove if $(a_n)$ is a decreasing sequence of real numbers and $\sum a_n$ converges then $\lim n a_n = 0$.}

        Note that $a_n \geq 0$. Assume $\exists i : a_i < 0 \implies \forall x > i: a_x < 0 \implies \lim a_n < 0$ which is a contradiction as $\sum a_n$ converges.

        Since $\sum a_n$ converges it must be Cauchy so $\forall \epsilon > 0 \exists m \in \N: \forall n > m: \abs{\sum_{k=m}^n a_k} < \epsilon$.

        Fix $\epsilon > 0$ then we know  $\exists x \in \N \forall n > x: \abs{\sum_{k=x}^n a_k} = a_x + \ldots + a_n < \frac{\epsilon}{4}$ since $\sum a_n$ converges.
        Similarily we know $\exists y \in \N \forall n > y: \abs{\sum_{k=x}^n a_k} = a_x + \ldots + a_n < \frac{\epsilon}{4x}$ since $\sum a_n$ converges.

        Then take $N = \max(x , y)$. Then $n a_n \leq m a_m + \sum_{k=m}^n a_k \leq \frac{\epsilon}{4} + \frac{\epsilon}{4} < \epsilon$.

        \textbf{ Use a to prove $\sum \frac{1}{n}$ diverges.}

        Assume $\sum \frac{1}{n}$ converges. Then a) implies that $\lim n \frac{1}{n} = 0$, but this is false. Therefore $\sum \frac{1}{n}$ must diverge.

    \item \textbf{ Write the following repeating decimals as rationals.}

        \begin{enumerate}
            \item $.2  = \frac{1}{5}$
            \item $.0\bar{2}  = \frac{.2}{9} = \frac{1}{45}$
            \item $.\bar{02}  = \frac{2}{99}$
            \item $3.\bar{14}  = 3 + \frac{14}{99} = \frac{311}{99}$
            \item $.\bar{10}  = \frac{10}{99} = \frac{10}{90}$
            \item $.1\bar{492}  = 1+ \frac{142}{999} = \frac{1141}{999}$
        \end{enumerate}

    \item \textbf{ Prove $f(x) = \sqrt x$ is continuous inside it's domain $Dom(f) = \{ x \in \R : x \geq 0 \}$}

        To prove continuity we will show $\forall x_0 \forall \epsilon > 0 \exists \delta > 0 : \abs{x - x_0} < \delta \implies \abs{f(x) - f(x_0)} < \epsilon$.

        Fix $x_0 \in Dom(f)$ and $\epsilon > 0 $. Take $\delta = \epsilon \sqrt{x_0}$.

        Then $\abs{x - x_0} = \abs{(\sqrt x - \sqrt x_0)(\sqrt x + \sqrt x_0)}  = \abs{\sqrt x - \sqrt x_0} \abs{\sqrt x + \sqrt x_0}   < \epsilon \sqrt x_0 = \delta.$
        
        But $\abs{\sqrt x - \sqrt x_0} \abs{\sqrt x_0} < \abs{\sqrt x - \sqrt x_0} \abs{\sqrt x_0}  < \epsilon \sqrt x_0 \implies \abs{\sqrt x - \sqrt x_0} < \epsilon$.

    \item \textbf{ Prove that $f$ is an injection with $Ran(f)$ being an interval in $\R$, and so $f^{-1}$ is defined.}

        Supose $f$ was not an injection, so $\exists x, y \in Dom(f): f(x) = f(y) \land x \neq y$. Then either $x < y$ or $y < x$.
        
        WLOG consider $x < y$. Then since $f$ is continuous and strictly increasing $\implies f(x) < f(y)$ which is a contradiction.  

        \textbf{ Prove $f^{-1}$ is continuous.}

        To prove continuity we will show $\forall x_0 \in Ran(f) \forall \epsilon > 0 \exists \delta > 0 : \abs{x - x_0} < \delta \implies \abs{f^{-1}(x) - f^{-1}(x_0)} < \epsilon$.

        Fix $x_0 \in Ran(f)$ and $\epsilon > 0 $. Take $\delta = \epsilon $.


    \item \textbf{Prove $f(x) = 1$ for $x > 0$ and $f(x) = 0$ for $x \leq 0 , x_0 = 0$ is discontinuous.} 

        Assume $f(x)$ was continuous at $x_0 = 0: \lim x_n = x_0 \implies  \lim f(x_n) = f(x_0)$ 
        But then take $x_n = \frac{1}{n}$.  $\lim x_n = 0$, but $\lim f(x) = 1 \neq f(0) = 0$. 

        \textbf{Prove $g(x) = sin (\frac{1}{x})$ for $x \neq 0$ and $g(x) = 0$ for $x = 0 , x_0 = 0$ is discontinuous.} 

        Assume $g(x)$ was continuous at $x_0 = 0: \lim x_n = x_0 \implies  \lim g(x_n) = g(x_0)$ 

        But then take $x_n = \frac{1}{2\pi n + \frac{\pi}{2}}$.  $\lim x_n = \lim \frac{1}{2\pi n + \frac{\pi}{2}} = 0$, but $\lim g(x)  = \lim sin(2\pi n + \frac{\pi}{2}) = sin(\frac{\pi}{2}) = 1 \neq g(0) = 0$. 

        \textbf{Prove $sgn(x) = -1 $ for $x < 0$, $sgn(x) = 0$ for $ x= 0$, $sgn(x) = 1 $ for $x >0, x+0=0$ is discontinuous.}

        Assume $sgn(x)$ was continuous at $x_0 = 0: \lim x_n = x_0 \implies  \lim sgn(x_n) = sgn(x_0)$ 
        But then take $x_n = \frac{1}{n}$.  $\lim x_n = 0$, but $\lim sgn(x) = 1 \neq sgn(0) = 0$. 


    \item \textbf{Prove if $\forall r \in \R: f(r) = 0 \implies \forall x \in (a, b): f(r) = 0$.}

        Take $x \in (a, b$). If $ x\ in \R$, then $f(r) = $ and we are done. If $x \in \I$ then by denseness we know that exists a sequence of rational numbers that converges to $x$. But $f$ is continuous so $lim r_n = x_0 \implies \lim f(r_n) = \lim f(x_0) = 0$.

        \textbf{Let $f, g$ be continuous real valued functions on $(a,b)$ such that $f(r) = g(r) \forall r\in \R$, prove $f(x) = g(x) \forall x$.}

        Let $ f-g = f(x) - g(x)$. Then since $\forall r \in \R \cap (a, b): (f-g)(x) = f(r) - g(r)  = 0$ by a) $ \forall x \in (a,b) : (f-g) (x) = 0  \implies f(x) - g(x) = 0 \implies f(x) = g(x)$.

    \item \textbf{Prove $\lim (a_n+b_n)$ exists and equals $A + B$}

        Fix $\epsilon$. Then for $\epsilon_1 = \frac{\epsilon}{3} \exists N \forall n > N: \abs{a_n - A} < \frac{epsilon}{3}$. And also for $\epsilon_2 = \frac{\epsilon}{3} \exists N \forall n > N: \abs{b_n - L} < \frac{\epsilon}{3}$.

        But then by triangle inequality $\abs{a_n + b_n - A -B} \leq \abs{a_n -A} +\abs{b_n - B} \leq \frac{2\epsilon}{3}$. So then $\lim (a_n + b_n) = A+B$.

        \textbf{Prove $\lim c (a_n)$ exists and equals $cA$}

        Note that $c (a_n) = \sum_{i=1}^c a_n$. But then by the sum rule above $\lim c(a_n) = \sum_{i=1}^c \lim a_n = cA$.

        \textbf{Prove $\lim (a_n b_n)$ exists and equals $AB$}

        Note $(a_n - A) (b_n -B) = a_n b_n - A( b_n) - B(a_n) + AB$. Solving for $a_n b_n = (a_n - A) (b_n-B) + A(b_n) + B(a_n) -AB$.

        Taking the sum rule yields $\lim a_n(b_n) = \lim (a_n -A)(b_n -B) + \lim A(b_n) + \lim B(a_n) + \lim {-AB}$
        
        But then we can apply the a to get that $\lim (a_n -A) = \lim a_n - \lim A = A -A = 0 \land \lim (b_n -B) = \lim b_n - \lim B = B -B = 0 $.

        so $\lim a_n(b_n) = 0  + AB  + BA - AB = AB $.

        \textbf{Prove $\lim \frac{a_n}{b_n} = \frac{A}{B}$, assuming $b_n, B \neq 0$. }

        First we'll prove $\lim \frac{1}{b_n} = \frac{1}{B}$ and then use the product rule. Fix $\epsilon > 0$. Take $\epsilon_1 = \frac{\abs{B}}{2}$ then $\exists N_1 \forall n > N_1 : \abs{b_n - B} < \frac{\abs{B}}{2}$.

        But then $\abs{B} = \abs{B - b_n +b_n} \leq \abs{B-b_n} + \abs{b_n} \leq \frac{\abs{B}}{2} + \abs{b_n} \implies \frac{1}{b_n} < \frac{2}{\abs{B}}$.

        Then take $\epsilon_2 = \frac{\abs{B}^2}{2} \epsilon$. $\exists N_2 \forall n > N_2: \abs{b_n - B} < \frac{\abs{B}^2}{2}\epsilon$.

        Then take $N = \min(N_1, N_2), \forall n > N: \abs{\frac{1}{b_n} - \frac{1}{B}} = \abs{\frac{B - b_n}{B(b_n)}} = \frac{1}{\abs{B}}\frac{1}{\abs{b_n}}\abs{B-b_n} < \frac{1}{\abs{B}} \frac{2}{\abs{B}} \frac{\abs{B}^2}{2}\epsilon < \epsilon$.

\end{enumerate}
\end{document}
