\documentclass[10pt,a4paper]{article}
\addtolength{\oddsidemargin}{-.875in}
\addtolength{\evensidemargin}{-.875in}
\addtolength{\textwidth}{1.75in}
\addtolength{\topmargin}{-.875in}
\addtolength{\textheight}{1.75in}

\usepackage{amsmath,amssymb}
\DeclareMathOperator*{\E}{\mathbb{E}}
\DeclareMathOperator*{\R}{\mathbb{R}}
\DeclareMathOperator*{\Q}{\mathbb{Q}}
\DeclareMathOperator*{\N}{\mathbb{N}}
\DeclareMathOperator*{\I}{\mathbb{I}}
\begin{document}

MATH 131 Homework 3

Jesse Cai

304634445

\begin{enumerate}
    \item \textbf{Let $a, b \in R$. Show if $a \leq b_1$ for every $b_1 > b$, then $a \leq b$.}

        Suppose $a > b$. Then by the denseness of $\Q$ (Thm 4.7) $\exists b_1 : a > b_1> b$ but this is a contradiction, as we said $a \leq b_1$ for every $b_1 > b$. Therefore $ a \leq b$ if $a \leq b_1$ for every $b_1 > b$.

    \item \textbf{Prove that for any $A,B \subset \R: \sup(A\cup B) = \max \{\sup(A), \sup(B)\}$.}

        WLOG Suppose $\sup(A) > \sup(B)$ then $\forall b \in B: \sup(B) > b \implies \forall b \in B: \sup(A) > \sup(B) > b$.
        
        So $\forall x \in A\cup B: \sup(A) > x \implies \sup(A)$ is an upper bound on $A \cup B$. Now we will prove that $\sup(A)$ is the least upper bound. 

        Suppose $\exists x : x \text{ is a upper bound} \land x < \sup A$. But then $x < \sup(A) \land \forall a \in A: x > a$. But this is a contradiction, as by definition $\sup(A)$ is the least upper bound. So therefore $\sup(A) = \sup (A\cup B)$.

        Note when $\sup(A) = \sup(B)$ either choice satisfies max.

    \item \textbf{Determine $\lim s_n$ where $s_n = \sqrt{n^2 +1} -n $ }

    \item \textbf{Determine $\lim s_n$ where $s_n = \sqrt{n^2 +1} -n $ }

    \item \textbf{Find $\lim \frac{4n+3}{7n-5}$ }

        Claim: $\lim \frac{4n+3}{7n-5} = \frac{4}{7}$

        Fix $k \in \N$. Then take $n_0 = \max (5, \frac{1}{k})$

        If $n \geq n_0$ then \frac{} -  \geq

        $ \lvert \frac{4n+3}{7n-5}- \frac{4}{7} \rvert = \frac{1}{k+1}$

    \item \textbf{Determine if $\lim_{n \to \inf}$ }
    \item \textbf{Determine if $\lim_{n \to \inf}$ }
    \item \textbf{Determine if $\lim_{n \to \inf}$ }
    \item \textbf{Determine if $\lim_{n \to \inf}$ }

\end{enumerate}
\end{document}
