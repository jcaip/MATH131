\documentclass[10pt,a4paper]{article}
\addtolength{\oddsidemargin}{-.875in}
\addtolength{\evensidemargin}{-.875in}
\addtolength{\textwidth}{1.75in}
\addtolength{\topmargin}{-.875in}
\addtolength{\textheight}{1.75in}

\usepackage{amsmath,amssymb}
\DeclareMathOperator*{\E}{\mathbb{E}}
\DeclareMathOperator*{\R}{\mathbb{R}}
\DeclareMathOperator*{\Q}{\mathbb{Q}}
\DeclareMathOperator*{\N}{\mathbb{N}}
\DeclareMathOperator*{\I}{\mathbb{I}}
\begin{document}

MATH 131 Homework 3

Jesse Cai

304634445

\begin{enumerate}
    \item \textbf{Let $a, b \in R$. Show if $a \leq b_1$ for every $b_1 > b$, then $a \leq b$.}

        Suppose $a > b$. Then by the denseness of $\Q$ (Thm 4.7) $\exists b_1 : a > b_1> b$ but this is a contradiction, as we said $a \leq b_1$ for every $b_1 > b$. Therefore $ a \leq b$ if $a \leq b_1$ for every $b_1 > b$.

    \item \textbf{Prove that for any $A,B \subset \R: \sup(A\cup B) = \max \{\sup(A), \sup(B)\}$.}

        WLOG Suppose $\sup(A) > \sup(B)$ then $\forall b \in B: \sup(B) > b \implies \forall b \in B: \sup(A) > \sup(B) > b$.
        
        So $\forall x \in A\cup B: \sup(A) > x \implies \sup(A)$ is an upper bound on $A \cup B$. Now we will prove that $\sup(A)$ is the least upper bound. 

        Suppose $\exists x : x \text{ is a upper bound} \land x < \sup A$. But then $x < \sup(A) \land \forall a \in A: x > a$. But this is a contradiction, as by definition $\sup(A)$ is the least upper bound. So therefore $\sup(A) = \sup (A\cup B)$.

        Note when $\sup(A) = \sup(B)$ either choice satisfies max.

    \item \textbf{Prove every nonempty set $F \subset P(E)$ admits $\sup F $ and $\inf F$ and that $\sup(F) = \cap F$}

        Let $F \subset P(E)$. Then take $\forall f \in F:  f \subset \cup F$ so $\cup F$ is an uppper bound. Suppose $\exists x \in P(E): (\forall f \in F: f\subset x) \land x \subset \cup F$. 
        But then this suggests $\exists a \in x  \text{ s.t. } a \not\in \cup F$ but this is a contradiction, as all elements in $x$ are in all sets of $f \implies \in \cup F$. So $\cup F$ is the least upper bound and thus the suprema.

        Likewise $\forall f \in F: \cap F \subset f$ so $\cap F$ is a lower bound. Suppose $\exists x \in P(E): (\forall f \in F: x \subset f) \land \cap F \subset x$. 
        But then this suggests $\exists a \in x a \not\in \cup F$ but this is a contradiction, as all elements in $x$ are in all sets of $f \implies \in \cap F$. So $\cap F$ is the least upper bound and thus the suprema.

        \textbf{Prove that $\lim_{n \to \infty} \inf A_n \subset \lim_{n \to \infty}\sup A_n$}

        $$\lim_{n \to \infty} \inf A_n = \bigcup_{i=0}^\infty \bigcap_{m=n}^\infty A_n$$
        $$\lim_{n \to \infty} \sup A_n = \bigcap_{i=0}^\infty \bigcup_{m=n}^\infty A_n$$

        Let $a \in \liminf A_n \implies \exists n \text{ s.t. } \forall m > n : a \in A_m $
        but then $a \in \bigcup_{m=n}^\infty A_m$. So we just need to show that this holds $\forall n$. Suppose there was a $n$ where this did not hold, so $a \not\in \bigcup_{m=n}^\infty A_m$. But if it is not in the union it cannot be in the intersection, so we have a contradiction and thus $a \in \limsup A_n$.


    \item \textbf{Determine $\lim s_n$ where $s_n = \sqrt{n^2 +1} -n $ }

        Claim: $\lim s_n  = 0$

        Proof: Fix $k \in \N$. Then take $n_0 = k+1$. 

        If $n \geq n_0$ then $\left\vert \sqrt{n^2 +1} -n \right\vert  = \left\vert \frac{(\sqrt{n^2 +1} -n)(\sqrt{n^2+1} +n)}{(\sqrt{n^2+1} +n)} \right \vert = \left\vert\frac{1}{\sqrt{n^2+1} + n} \right \vert$

        But note that $\sqrt{n^2+1} \geq n$ so $\left\vert\frac{1}{\sqrt{n^2+1} + n} \right \vert  = \frac{1}{\sqrt{n^2+1} + n} < \frac{1}{2n} < \frac{1}{n} < \frac{1}{n_0} = \frac{1}{k+1}$

    \item \textbf{Find $\lim \frac{4n+3}{7n-5}$ }

        Claim: $\lim \frac{4n+3}{7n-5} = \frac{4}{7}$

        Proof: Fix $k \in \N$. Then take $n_0 = \max (5, {1+k})$

        If $n \geq n_0$ then $ \lvert \frac{4n+3}{7n-5}- \frac{4}{7} \rvert = \frac{7(4n+3) - 4(7n-5)}{7(7n-5)} = \frac{41}{7(7n-5)}$
        but since $n \geq 5 \implies \frac{41}{7(7n-5)} < \frac{1}{n} < \frac{1}{1+k}$. So this is indeed a limit.


    \item \textbf{Find $\lim \frac{1}{n}\sin(x)$ }

        Claim: This limit is 0

        Proof: Fix $k$. Then take $n_0 = 1+k$. 

        If $n \geq n_0$ then $\lvert \frac{1}{n}\sin(x) \rvert = \lvert \frac{1}{n} \rvert \lvert \sin(x) \rvert < \lvert \frac{1}{n} \rvert = \frac{1}{n} <  \frac{1}{n_0} = \frac{1}{k+1}$


    \item \textbf{Prove $\lim s_n = 0$ iff $\lim \lvert s_n \rvert = 0$.}

        By definition we say that $\lim s_n = 0$  if 

        $$\forall k \in \N \exists n_0 \forall n \in \N : n \geq n_0 \implies \lvert s_n - 0\rvert < \frac{1}{k}$$

        But note that $\lvert \lvert s_n \rvert \rvert  = \lvert s_n \rvert$ so via subsitution
        $$\left [ \forall k \in \N \exists n_0 \forall n \in \N : n \geq n_0 \implies \lvert s_n - 0\rvert < \frac{1}{k} \right ] \implies \text{ the limit for $\lvert \lvert s_n \rvert \rvert = 0$. } $$        
        and likewise
        $$\left [ \forall k \in \N \exists n_0 \forall n \in \N : n \geq n_0 \implies \lvert \lvert s_n - 0 \rvert \rvert < \frac{1}{k} \right ] \implies \text{ the limit for $\lvert s_n \rvert = 0$. } $$

        \textbf{Observer for $s_n = (-1)^n$ $\lim \lvert s_n \rvert = 0$ but $\lim s_n = 0$ does not exist.}

        $\lim \lvert s_n \rvert  = 1$, as $\lvert (-1) ^ n \rvert$ is always equal to $1$. However, $s_n$ will oscillate between $-1$ and $1$ indefinitely, and as we showed in class, a sequence can only have one value as a limit, so therefore the limit does not exist.

    \item \textbf{Find $\lim \sqrt[3]{n^3+n^2+1} - \sqrt[3]{n^3+1}$}

        Claim: This limit is 0

        Proof: Let $a = \sqrt[3]{n^3+n^2+1}$ and $b = \sqrt[3]{n^3+1}$. Fix $k$. Then take $n_0  = k+1$.

        If $n \geq n_0$ then $\lvert a-b \rvert = \lvert \frac {(a-b)(a^2+ab+b^2)}{a^2+ab+b^2} \rvert  = \frac{n^2}{a^2+ab+b^2} $
        
        But since $a,b> n \implies ab > n^2, aa> n^2, bb > n^2$:
        
        $$\frac{n^2}{a^2+ab+b^2} < \frac{n^2}{3n^2} < \frac{n^2}{n^3} = \frac{1}{n} < \frac{1}{n_0} = \frac{1}{k+1}$$

    \item \textbf{Determine if $\{a_n\}_{n \in \N}$ exists.}

        $$ a_0 := 1 \land \left ( \forall n \in \N: a_{n+1} = \sqrt{2+a_n}\right)$$

        Claim: The limit is $2$

        Proof: First by induction, we will show that $\{ a_n \}$ is increasing. So let $P(n) = a_{n+1} \geq a_{n}$

        Base case:  We can see that $a_1 = \sqrt{3} \geq a_0 = 1$ so this is TRUE by inspection.

        Inductive Step: Assume $P(n)$ and we will consider $P(n+1)$

        If $a_{n+1} \geq a_{n}$ then $\sqrt{2 + a_{n+1}} \geq \sqrt{2 + a_n}  \implies a_{n+2} > a_{n+1} \implies P(n+1)$

        Next we will show that $\{ a_n \}$ is bounded, again via induction. Let $P(n) = a_n < 3$

        Base Case: $a_0 = 1 < 3$ so $P(0)$ is TRUE.
        Inductive Step: Assume $P(n)$ the consider $P(n+1)$
        $a_n < 3 \implies \sqrt{2+a_n} < 3$ since at most on the LHS we can get is $\sqrt3 \implies a_{n+1} < 3$

        But since $f(x) = \sqrt{2+ x}$ is continuous:

        $$ L = \lim_{n \to \inf} a_{n+1} = \lim_{n \to \inf}\sqrt{2+a_n} = \sqrt{2+\lim_{n \to\ inf}a_n}  = \sqrt{2+L}$$

        Solving this equation we get $L^2 = 2+L$ so $L = 2$.




\end{enumerate}
\end{document}
